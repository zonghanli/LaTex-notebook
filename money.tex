\documentclass[oneside,a4papar,11pt,UTF8]{report}
\usepackage[utf8]{inputenc}
\usepackage[UTF8]{ctex}
\usepackage{amsmath}
\usepackage{mathpazo}  
\usepackage{algorithm}  
\usepackage{algorithmicx}  
\usepackage{algpseudocode}  
\floatname{algorithm}{算法}  
\renewcommand{\algorithmicrequire}{\textbf{输入:}}  
\renewcommand{\algorithmicensure}{\textbf{输出:}}
\def\angles#1{{%
		\vbox{\hrule height .2pt
			\kern 1pt
			\hbox{$\scriptstyle {#1}\kern 1pt$}%
		}\kern-.05pt \vrule width .2pt
}}
%
\begin{document}
	\title{保险精算}
	\author{OscarLi}
	\maketitle 
	\tableofcontents
	\chapter{利息度量}
	
	\section{贴现}
	$v=\frac{1}{1+i}$
	\chapter{等额年金}
	\noindent$a_{\angles{n}},s_{\angles{n}}$\\	$\ddot{a}_{\angles{n}},\ddot{s}_{\angles{n}}$
	\section{期初期末年金的现值}
	$a_{\angles{n}}=v+v^2+v^3+\dots+v^n$
\\ $\ddot{a}=\frac{1-v^n}{d}$
\chapter{变额年金}

\chapter{收益率}
\end{document}
