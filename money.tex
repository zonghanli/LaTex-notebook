\documentclass[cn,11pt,chinese,black]{elegantbook}
\title{利息理论}
\subtitle{made by \LaTeX{} }
\author{OscarLi}
\institute{西南交大数学学院}
\cover{cover14.pdf}
% 本文档命令
\usepackage{array}
\newcommand{\ccr}[1]{\makecell{{\color{#1}\rule{1cm}{1cm}}}}
\begin{document}
\def\angles#1{{%
		\vbox{\hrule height .2pt
			\kern 1pt
			\hbox{$\scriptstyle {#1}\kern 1pt$}%
		}\kern-.05pt \vrule width .2pt
}}
%
\maketitle
	\chapter{利息度量}
	
	\section{贴现}
	$v=\frac{1}{1+i}$
	\chapter{等额年金}
	\noindent$a_{\angles{n}},s_{\angles{n}}$\\	$\ddot{a}_{\angles{n}},\ddot{s}_{\angles{n}}$
	\section{期初期末年金的现值}
	$a_{\angles{n}}=v+v^2+v^3+\dots+v^n$
\\ $\ddot{a}=\frac{1-v^n}{d}$
\chapter{变额年金}

\chapter{收益率}
\end{document}
