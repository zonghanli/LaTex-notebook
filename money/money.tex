\documentclass[cn,11pt,chinese,black]{elegantbook}
\title{利息理论}
\subtitle{made by \LaTeX{} }
\author{OscarLi}
\institute{西南交大数学学院}
\cover{cover14.pdf}
% 本文档命令
\usepackage{array}
\newcommand{\ccr}[1]{\makecell{{\color{#1}\rule{1cm}{1cm}}}}
\begin{document}
\def\angles#1{{%
		\vbox{\hrule height .2pt
			\kern 1pt
			\hbox{$\scriptstyle {#1}\kern 1pt$}%
		}\kern-.05pt \vrule width .2pt
}}
%
\maketitle
	\chapter{利息度量}
	
	\section{贴现}
	$v=\frac{1}{1+i}$
	\chapter{等额年金}
	\noindent$a_{\angles{n}},s_{\angles{n}}$\\	$\ddot{a}_{\angles{n}},\ddot{s}_{\angles{n}}$
	\section{期初期末年金的现值}
	$a_{\angles{n}}=v+v^2+v^3+\dots+v^n$
\\ $\ddot{a}=\frac{1-v^n}{d}$
\chapter{变额年金}
\begin{definition}{期末付递增年金}
	$(Ia)_{\angles{n}}=v+2 v^{2}+3 v^{3}+\cdots+(n-1) v^{n-1}+n v^{n}$
\end{definition}
\begin{remark}
	$(Ia)_{\angles{n}}=\frac{\ddot{a}_{\angles{n}}-n v^{n}}{i}$
\end{remark}
$(Is)_{\angles{n}}=(1+i)^n(Ia)_{\angles{n}}$
\chapter{收益率}
\begin{definition}{Dollar-Weighted Return For a One-Year Period}
\noindent	Suppose the following information is known:
	(i) the balance in a fund at the start of the year is $A$ \\
	(ii) for $0<t_{1}<t_{2}<\cdots<t_{n}<1,$ the net deposit at time $t_{k}$ is amount $C_{k}$ (positive for a net deposit, negative for a net withdrawal), and \\
	(iii) the balance in the fund at the end of the year is $B$
	Then the net amount of interest earned by the fund during the year is $I=B-\left[A+\sum_{k=1}^{n} C_{k}\right],$ and the dollar-weighted rate of return earned by the fund for the year is
	\[
	\frac{I}{A+\sum_{k=1}^{n} C_{k}\left(1-t_{k}\right)}
	\]
\end{definition}
\begin{remark}
	(I a)_{\bar{n}}=\frac{\ddot{a}_{\bar{n} 1}-n v^{n}}{i}
\end{remark}
\begin{definition}{Time-Weighted Return For a One-Year Period}
\noinent Suppose the following information is known:\\
	(i) the balance in a fund at the start of the year is $A$\\
	(ii) for $0<t_{1}<t_{2}<\cdots<t_{n}<1,$ the net deposit at time $t_{k}$ is amount $C_{k}($ positive for a net deposit, negative for a net withdrawal)\\
	(iii) the value of the fund just before the net deposit at time $t_{k}$ is $F_{k},$ and\\
	(iv) the balance in the fund at the end of the year is $B$
	The time-weighted return rate earned by the fund for the year is
	\[
	\left[\frac{F_{1}}{A} \times \frac{F_{2}}{F_{1}+C_{1}} \times \frac{F_{3}}{F_{2}+C_{2}} \times \cdots \times \frac{F_{k}}{F_{k-1}+C_{k-1}} \times \frac{B}{F_{k}+C_{k}}\right]-1
	\]	
\end{definition}	
\section{再投资}
\end{document}
