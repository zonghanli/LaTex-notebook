\documentclass[cn,11pt,chinese,black]{elegantbook}
\title{Topology}
\subtitle{made by \LaTeX{} }
\author{OscarLi}
\institute{西南交大数学学院}
\cover{cover14.pdf}
% 本文档命令
\usepackage{array}
\newcommand{\ccr}[1]{\makecell{{\color{#1}\rule{1cm}{1cm}}}}
\begin{document}
\maketitle
\chapter{度量空间}
\section{绝对值}
\noindent distance$:|a-b|$ \\
properties$:(1)|x| \geq 0$,for all $x \in R$,and $"=” \Leftrightarrow x=0$\\
$(2):|a-b|=|b-a|(|x|=|-x|)$\\
$(3):|x+y| \leq |x|+|y|$,for all $x,y \in R$\\
($|a-c| \leq |a-b|+|b-c|$)
\section{度量空间}
\noindent Distance function/metric space\\
Let $X$ be a set.
$\underline{Def:}$A function $X \times X \stackrel{d}{\longrightarrow}\mathbb{R}$is called a distance function on $X$
1.$\forall x,y\in X$,$d(x,y)\geq 0$ and $"=” \Leftrightarrow x=y$\\
2.$\forall x,y\in X$,$d(x,y)=d(y,x)$\\
3.$\forall x,y,z \in X$,$d(x,z)\leq d(x,y)+d(y,z)$\\
\begin{example}
$\mathfrak{A}:$ \\
1.$x=(x_1,x_2,\dots,x_m),y=(y_1,y_2,\dots,y_m)\in \mathbb{R}^n$\\
$d_2(x,y):=\sqrt{|x_1-y_1|^2+\cdots+|x_m-y_m|^2}=|x-y|$ \\
$d_2$ is a metric on $\mathbb{R}^n$(Cauchy inequality)\\
2.$d_1(x,y):=|x_1-y_1|+|x_2-y_2|+\cdots+|x_m-y_m|$\\
3.$d_{\infty}(x,y)=max\{|x_1-y_1|,\dots,|x_m-y_m|\}$
\end{example}
\begin{example}
$\mathfrak{B}:$ \\
X:a set.For $x,y \in X$,let $$d(x,y):=\left\{
\begin{aligned}
1&if&x\leq y
\\
0&if&x =y
\end{aligned}
\right.
$$ 
$d(x,y)\Rightarrow$the discrete metric
\end{example}
\section{开集,闭集}
\begin{definition}{开集}
\noindent Let $(X, \mathcal{T})$ be any topological space. Then the members of $\mathcal{T}$ are said to be open sets.
\end{definition}
\begin{remark}
	开集和闭集不是一个相对概念,同一个成员可能同时开同时闭
\end{remark}
\begin{example}
	$\begin{array}{lll}  \text { Let } X=\{a, b, c, d, e, f\} \text { and }\end{array}$
	\[
	\tau_{1}=\{X, \emptyset,\{a\},\{c, d\},\{a, c, d\},\{b, c, d, e, f\}\}
	\]
	找出又开又闭的集合,只开,只闭的集合
\end{example}
\begin{remark}
	
\end{remark}
\begin{definition}{有限补拓扑}
	$\mathcal{T}_{f}=\left\{A \subset X | \quad A^{c} \text { 是 } X \text {的有限子集}\right\} \cup\{\emptyset\}$
\end{definition}
\begin{definition}{可数补拓扑}
	$\mathcal{T}_{c}=\left\{A \subset X | \quad A^{c} \text { 是 } X \text {的可数子集}\right\} \cup\{\emptyset\}$
\end{definition}
\noindent we may generalize the definitions about limits and convergence to metric space \\
$\underline{Def}$ Let $(X,d)$ be a metric space,$a_n(n \in \mathbb{N})$be a seq in $\mathrm{X}$.and $\mathcal{L}$in X\\
$a_n(n \in \mathbb{N})$converges to $\mathcal{L}$\\
(1)For $r \geq 0$and $x_0 \in X$,we let $B_r(x_0)=\{x \in X|d(x,x_0)\leq r\}$(open ball)\\
(2).S is an open set(of$(X,d)$),if $\forall x \in S$,$\exists r >0$
($B_r(x_0)\subset S$)open ball $\Rightarrow$open set \\
\begin{exercise}
$(X,d):$metric space.$x_0 \in X,r \geq 0$
Show that:\\
(1)$B_r(x_0)$is open \\
(2)$\{x \in X|d(x,x_0)> r\}$is open
\end{exercise}
\chapter{拓扑空间}
\noindent$\underline{Def:}$A topology space \\
 $\mathcal{X}=(\underline{X},\eth_{x})$consists of a set $\underline{X}$,called the underlying space of $\mathcal{X}$ ,and a family $\eth_{x}$of subsets of $\mathcal{X}$(ie.$\eth_{x}\subset P(\underline{X})$)
 $P(\underline{X})$means the power set of $\underline{X}$\\
 s.t.:(1):$\underline{X}$ and $\varnothing \in \eth_{x}$\\
 (2):$U_{\alpha}\in \eth_{x}(\alpha \in A) \Rightarrow$
 $\cup_{\alpha \in A}U_{\alpha} \in \eth_{x}$\\
 (3).$U,U^{\prime}\in \eth_{x} \Rightarrow U \cap U^{\prime} \in \eth_{x}$
 $\eth_{x}$ is called a topology(topological structure) on $\underline{X}$\\
$\underline{Convention:}$We usually use $\mathcal{X}$ to indicate the set $\underline{X}$and omit the subscript $x$ in $\eth_{x}$ by saying "a topological space$(X,\eth)$"\\
$\underline{Examples:}$(1)metric space:
$(X,d) \looparrowright(X,\eth_{d})$(open sets induced by d)\\
$\bullet$Different distance funcs might determine the same topology
\begin{definition}{内点,内部}
\noindent	 已知拓扑空间 $X, x \in X, A \subseteq X,$ 若
	\[
	\exists x \text { 的邻域 } U, \text { s.t. } \quad U \subseteq A
	\]
	则称 $x$ 是 $A$ 的 内点 $. A$ 的全体内点记为 $A^{\circ},$ 称为 $A$ 的内部 $(\text { interior })$
\end{definition}
\begin{property}
(1). 若 $A \subset B,$ 则 $A^{\circ} \subset B^{\circ}$ \\
(2). $A^{\circ}=\bigcup\{U \subset X | U \text { 是 } X \text { 的开集且 } U \subset A\},$ 因此 $A$ 的内部 $A^{\circ}$ 是包含在
$A$ 中最大的开集;\\
(3). $A$ 是开集 $\Longleftrightarrow A=A^{\circ}$\\
$(4) \cdot(A \cap B)^{\circ}=A^{\circ} \cap B^{\circ}$ \\
$(5) .(A \cup B)^{\circ} \supset A^{\circ} \cup B^{\circ}$ \\
\end{property}
\end{document}
